\section{Java™}

\lipsum[2-3]

\begin{lstlisting}[language=Java, caption={Hola Mundo en Java™}]
public class Example{
    public static void main(String[] args){
        System.out.println("Hola Mundo");
    }
}
\end{lstlisting}

\lipsum[1-2]

\section{Python}

\lipsum[1-2]

\begin{figure}[H]
    \centering
    \includegraphics[width=0.5\textwidth]{python-logo}
    \caption{El logotipo de Python}
    \label{fig_1}
\end{figure}

\lipsum[2-3]

\separation
\begin{alertblock}{¡Error de principiante!}
    \lipsum[4-5]

    \separation
\begin{lstlisting}[language=Python, caption={Error en Python}]
number_one = 12
number_two = number_one
number_one +=1
print(number_two) # Output: 12

list_a = [1,2,3,4]
list_b = list_a
list_a += [5]
print(list_b)  #Output: [1,2,3,4,5]
\end{lstlisting}

\end{alertblock}

\separation
\lipsum[2-3]

\separation
\begin{infoblock}{Python 2 está deprecado}
En enero de 2020 termino el soporte para \textit{Python 2} dejandolo oficialmente deprecado\footnote{Fuente: \href{https://www.python.org/doc/sunset-python-2/}{https://www.python.org/doc/sunset-python-2/}}.
\end{infoblock}

\separation
\lipsum[1-2]

\separation
\begin{infoblock*}{Un información no tan importante}
$a = a + 1$ \\
Es equivalente a: \\
$a += 1$
\end{infoblock*}